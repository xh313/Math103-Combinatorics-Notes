\documentclass[12pt]{article}
% Some custom commands to make typing easier. Feel free to add to it


% quick homomorphism
\newcommand{\homo}[1]{\varphi (#1)}
% quick inverse
\newcommand{\inv}[1]{#1^{-1}}
% quick generating set
\newcommand{\genset}[1]{\langle #1 \rangle}
% split problem into multiple parts
\newcommand{\spl}{\rule{\textwidth}{0.4pt}}
% quick normal subgroup
\newcommand{\nsub}{\trianglelefteq}
% quick Z/nZ
\newcommand{\znz}[1]{\mathbb{Z\text{/#1}Z}}

% quick projective surface
\newcommand{\proj}[1]{\mathbb P^{#1}(\mathbb C)}
% quick elliptic curve
\newcommand{\ellip}{E(\mathbb{C})}
% quick ramification
\newcommand{\ram}[1]{e_{#1}}

% todo list
\usepackage[color=yellow]{todonotes}
\newcommand{\tbd}[1]{\todo[inline]{#1}
\addcontentsline{toc}{subsubsection}{TODO}}

% scaling tikz
\usepackage{adjustbox}

% format bullets
% \usepackage[shortlabels]{enumitem}

% Document inherent packages to be loaded
\usepackage{amsmath, amsfonts, amssymb, amsthm, graphicx, url, xcolor, enumerate, tcolorbox}

%geometry helps manage margins, among other things.
\usepackage[rmargin=2in,lmargin=1in,tmargin=1in,bmargin=1in]{geometry}
\newcommand{\sidenote}[1]{\marginpar{\footnotesize \begin{itemize}
    \item[$\leftarrow$]\raggedright #1
\end{itemize}}}

\setlength{\headheight}{16pt}
\setlength{\marginparwidth}{1.5in}
\makeatletter      % make title smaller   
\def\@maketitle{   % custom maketitle 
\begin{center}
    {\Large \@title} \\[0.1in] \@author \\ {\small \@date}
\end{center}  
}
\setcounter{secnumdepth}{0}

% new color for the hypers
\definecolor{hyperpurp}{RGB}{108,113,196}
\usepackage[colorlinks = true,
            linkcolor = hyperpurp,
            urlcolor  = hyperpurp,
            citecolor = hyperpurp,
            anchorcolor = hyperpurp]{hyperref}

\usepackage[all,2cell,ps]{xy}
\usepackage{tikz, tikz-cd}
\usepackage{faktor}
\allowdisplaybreaks
\graphicspath{{Images/}}

\theoremstyle{definition}  % to get rid of the italics
\newtheorem{theorem}{\bt{Theorem}}%[section]
\newtheorem{proposition}[theorem]{\bt{Proposition}}
\newtheorem{corollary}[theorem]{\gt{Corollary}}
\newtheorem{lemma}[theorem]{\bt{Lemma}}
\newtheorem{conjecture}[theorem]{Conjecture}

\newtheorem{defn}{\rt{Definition}}

\newtheorem{eg}{\textcolor{violet}{Example}}
\newtheorem{noneg}[eg]{\textcolor{violet}{Non-example}}

\newtheorem*{rmk}{Remark}


%Import the natbib package and sets a bibliography  and citation styles
\usepackage{natbib}

% Create headers
\usepackage{fancyhdr}
\usepackage{xcolor}
\renewcommand{\headrulewidth}{0pt}
\fancypagestyle{updated}{%
    \fancyhead{MATH103 Notes \hfill\hfill Xuehuai He}
    \fancyfoot{\hyperlink{toc}{Back to TOC}\hfill\thepage \hfill \textcolor{lightgray}{\today}}
}

% \newcommand{\Belyi}{Bely\u{\i}~}
% \newcommand{\thm}[3]{\vskip 0.2in \begin{center} \fbox{\begin{minipage}{0.8\linewidth} \begin{theorem}[#1] \label{#2} #3 \end{theorem} \end{minipage}} \end{center} \vskip 0.2in}
% \newcommand{\prop}[3]{\vskip 0.2in \begin{center} \fbox{\begin{minipage}{0.8\linewidth} \begin{proposition}[#1] \label{#2} #3 \end{proposition} \end{minipage}} \end{center} \vskip 0.2in}
% \newcommand{\cor}[3]{\vskip 0.2in \begin{center} \fbox{\begin{minipage}{0.8\linewidth} \begin{corollary}[#1] \label{#2} #3 \end{corollary} \end{minipage}} \end{center} \vskip 0.2in}
% \newcommand{\conj}[3]{\vskip 0.2in \begin{center} \fbox{\begin{minipage}{0.8\linewidth} \begin{conjecture}[#1] \label{#2} #3 \end{conjecture} \end{minipage}} \end{center} \vskip 0.2in}

\DeclareMathOperator{\Aut}{Aut}
\DeclareMathOperator{\Gal}{Gal}
\DeclareMathOperator{\Mon}{Mon}
\DeclareMathOperator{\lcm}{lcm}


\newcommand{\rt}[1]{\textcolor{magenta}{#1}}
\newcommand{\gt}[1]{\textcolor{teal}{#1}}
\newcommand{\bt}[1]{\textcolor{cyan}{#1}}

\newcommand{\N}{\mathbb{N}}
\newcommand{\Z}{\mathbb{Z}}
\newcommand{\C}{\mathbb{C}}
\newcommand{\R}{\mathbb{R}}
\newcommand{\Q}{\mathbb{Q}}
\newcommand{\F}{\mathbb{F}}

\newcommand{\ifnif}{\textit{if and only if }}

\newcommand{\addlink}[1]{\addcontentsline{toc}{subsubsection}{#1}}

\newcommand{\circled}[1]{\begin{tikzpicture}[baseline=(word.base)]
\node[draw, rounded corners, text height=8pt, text depth=2pt, inner sep=2pt, outer sep=0pt, use as bounding box] (word) {#1};
\end{tikzpicture}
}

\newcommand{\ontangent}[1]{

    \spl

\textcolor{darkgray}{\textit{(Scratch work begins)}}
{#1}
\textcolor{darkgray}{\textit{(Scratch ends here)}}

\spl

}

\usepackage{ulem}
\usepackage{float}
\usepackage{libertinus}

% Make enumeration better
\usepackage{enumitem}
\setlist[itemize]{parsep=0pt,topsep=0pt}
\setlist[enumerate]{parsep=0pt,topsep=0pt}

% \definecolor{pomona}{HTML}{#0057b8}

\parskip = .2in
\parindent = 0in

% clever ref (load last)
\usepackage[capitalise]{cleveref}
\newcommand{\fref}{\cref}
\newcommand{\Fref}{\Cref}
\newcommand{\prettyref}{\cref}
\newcommand{\newrefformat}[2]{}

 %Cleveref definitions
\crefname{lemma}{Lemma}{Lemmas}
\crefname{thm}{Theorem}{Theorems}
\crefname{defn}{Definition}{Definitions}
\crefname{notn}{Notation}{Notations}
\crefname{const}{Construction}{Constructions}
\crefname{prop}{Proposition}{Propositions}
\crefname{rem}{Remark}{Remarks}
\crefname{cor}{Corollary}{Corollaries}
\crefname{equation}{Equation}{Equations}
\crefname{ex}{Example}{Examples}

%=============

\begin{document}
\title{MATH103 Combinatorics Notes}
\author{Xuehuai He}
\maketitle

\hypertarget{toc}{}
{\parskip=0.05in
\tableofcontents}

\spl

\newpage
\pagestyle{updated}
\section{Combinatorics}
\subsection{A1 Intro}
\rmk Let there be a set $\{1,2,\dots,n\}$. The number of subsets of it is $2^n$ since for each number, we could say ``include'' or ``exclude''.

\eg Now consider the number of subsets with no two adjacent elements. Call them \textit{good} subsets, and the count be $f(n)$.

\ontangent{

    First consider $n=0$. Then the only \textit{good} subset is $\emptyset$.
    
    Now consider $n=1$, both $\emptyset, \{1\}$ are good.

    Now consider $n=2$. We have subsets: $\emptyset, 1, 2, 12$\sidenote{notation simplified for fast typing}. The set 12 is not good.

    Similarly, we have $f(3)=5, f(5)=8$.

}

We have $f(n)=f(n-1)+f(n-2)$ for all $n\geq 2$. Hence, $f(n)$ is the sequence that satisfies the recurrence relation and the initial conditions $f(0)=1, f(1)=2$.

\subsection{A2 Fibonnacci Sequence}
\begin{align*}
    0,1,1,2,3,5,8,13,21,34,55,89,144,233,377,\dots
\end{align*}

\rmk Two notation conventions: \begin{itemize}
    \item $F_0=1, F_1=1, F_n=F_{n-1}+F_{n-2}\quad \forall n\geq 2$, and\sidenote{Textbook}
    \item $f_0=0, f_1=1, f_n=f_{n-1}+f_{n-2}\quad \forall n\geq 2$.\sidenote{Preferred!}
\end{itemize}

\begin{table}[htbp]
    \centering
    \caption{Table of the sequence in two notations}
    \begin{tabular}{cccccccccc}
        $n$   & 0& 1& 2& 3& 4& 5& 6& 7& 8\\
        $F_n$ & 1& 1& 2& 3& 5& 8& 13& 21& 34\\
        $f_n$ & 0& 1& 1& 2& 3& 5& 8& 13& 21
    \end{tabular}
\end{table}\sidenote{It is the same recurrence as A1 but with init conditions shifted: $f(n)=F_{n+1}=f_{n+2}$.}

\eg Prof Rad is climbing 47 steps. Energized by coffee, she sometimes climbds one step per stride, sometimes two steps per stride. In how many ways can she do this?

\ontangent{Let $S(n)$ be the number of ways climbing $n$ steps.
\begin{itemize}
    \parskip=0.2in
    \item $S(1)=1$\hfill {\begin{tikzcd}[ampersand replacement=\&,cramped,sep=small]
        \bullet \& \bullet
        \arrow[no head, from=1-1, to=1-2]
    \end{tikzcd}}
    \item $S(2)=2$\hfill \begin{tikzcd}[ampersand replacement=\&,cramped,column sep=small,row sep=tiny]
        \bullet \& \bullet \& \bullet \\
        \bullet \&\& \bullet
        \arrow[no head, from=1-1, to=1-2]
        \arrow[no head, from=1-2, to=1-3]
        \arrow[no head, from=2-1, to=2-3]
    \end{tikzcd}
    \item $S(3)=3$ \hfill \begin{tikzcd}[ampersand replacement=\&,cramped,column sep=small,row sep=tiny]
        \bullet \& \bullet \& \bullet \& \bullet \\
        \bullet \&\& \bullet \& \bullet \\
        \bullet \& \bullet \&\& \bullet
        \arrow[no head, from=1-1, to=1-2]
        \arrow[no head, from=1-2, to=1-3]
        \arrow[no head, from=2-1, to=2-3]
        \arrow[no head, from=1-3, to=1-4]
        \arrow[no head, from=2-3, to=2-4]
        \arrow[no head, from=3-1, to=3-2]
        \arrow[no head, from=3-2, to=3-4]
    \end{tikzcd}
    \item $S(4)=5$ \hfill \begin{tikzcd}[ampersand replacement=\&,cramped,column sep=small,row sep=tiny]
        \bullet \& \bullet \& \bullet \& \bullet \& \bullet \\
        \bullet \&\& \bullet \& \bullet \& \bullet \\
        \bullet \& \bullet \&\& \bullet \& \bullet \\
        \bullet \& \bullet \& \bullet \&\& \bullet \\
        \bullet \&\& \bullet \&\& \bullet
        \arrow[no head, from=1-1, to=1-2]
        \arrow[no head, from=1-2, to=1-3]
        \arrow[no head, from=2-1, to=2-3]
        \arrow[no head, from=1-3, to=1-4]
        \arrow[no head, from=2-3, to=2-4]
        \arrow[no head, from=3-1, to=3-2]
        \arrow[no head, from=3-2, to=3-4]
        \arrow[no head, from=1-4, to=1-5]
        \arrow[no head, from=2-4, to=2-5]
        \arrow[no head, from=3-4, to=3-5]
        \arrow[no head, from=4-1, to=4-2]
        \arrow[no head, from=4-2, to=4-3]
        \arrow[no head, from=4-3, to=4-5]
        \arrow[no head, from=5-1, to=5-3]
        \arrow[no head, from=5-3, to=5-5]
    \end{tikzcd}
\end{itemize}
Conjecture: maybe Fibonnacci?

}
\begin{proof}
    Consider the set of ways she can cover $n$ steps. We have two cases:\begin{enumerate}
        \item Her first stride is 1 step. Then, the number of ways is the number of ways to cover the remaining $n-1$ steps. Thus, this gives us $S(n-1)$ ways.
        \item Her first stride is 2 steps. Then the number of ways is the number of ways to cover the remaining $n-2$ steps. Thus, this gives us $S(n-2)$ ways.
    \end{enumerate}
    Therefore, we conclude that $S(n)=S(n-1)+S(n-2)$. We account the initial conditions and conclude the closed form: \begin{align*}
        S(n) = F_n = f_{n+1}
    \end{align*}
    for all $n$. Since Prof Rad climbs 47 steps, we get \circled{$S(47)=4807526976$}.
\end{proof}

\subsection{A3 Simplex numbers}
\defn \textit{Two-dimensional} triangular numbers: $T_2(n)=1+2+3+\dots+n$
\begin{itemize}
    \item $T_2(1)=1$\hfill \begin{tikzcd}[ampersand replacement=\&,cramped,sep=tiny]
        \bullet
    \end{tikzcd}
    \item $T_2(2)=1+2=3$\hfill \begin{tikzcd}[ampersand replacement=\&,cramped,sep=tiny]
        \& \bullet \\
        \bullet \& \bullet
        \arrow[no head, from=1-2, to=2-2]
        \arrow[no head, from=1-2, to=2-1]
        \arrow[no head, from=2-1, to=2-2]
    \end{tikzcd}
    \item \dots
\end{itemize}
\begin{align*}
    1,3,6,10,15,21,28,36,45,55,\dots
\end{align*}
\addlink{Triangular numbers}

\begin{theorem}
    $T_2(n)=1+2+\dots+n=\frac{n(n+1)}{2}$
\end{theorem}
\begin{proof}[First proof]We prove by induction.\sidenote{Not as good of a proof: we must know what we are proving in the first place!}
    \begin{itemize}[align=left]
        \item[Base case $n=1$:] $T_2(1)=1$, formula gives $\frac{1(1+1)}{2}=1$.
        \item[Inductive hypothesis:] Suppose proved formula for up to $n=k$.
        \item[Inductive step:] Consider $n=k+1$. \begin{align*}
            T_2(k+1)&=1+\dots+k+(k+1)\\
            &= T_2(k)+k+1\\
            &= \frac{k(k+1)}{2}+k+1\\
            &= \frac{k^2+k+2(k+1)}{2}\\
            &= \frac{k^2+3k+2}{2}\\
            &= \frac{(k+1)(k+2)}{2}\\
            &= \frac{(k+1)\left((k+1)+1\right)}{2}
        \end{align*}  
    \end{itemize}
\end{proof}

\begin{proof}[Proof by Gauss]
    Observe:\sidenote{Better proof: concluding the formula without knowing it first!}
    \begin{align*}
        T_2(n) & = 1+2 +\dots + (n-1) + n\\
        &= n+(n-1) + \dots + 2 + 1
    \end{align*}
    Therefore, we \textbf{add} the two rows:
    \begin{align*}
        2T_2(n)&= \underset{n}{\underbrace{(n+1)+(n+1)+\dots+(n+1)}}\\
        &= n(n+1)\\
        \therefore T_2(n) = \frac{1}{2}n(n+1)
    \end{align*}
\end{proof}

\defn Tetrahedral numbers: $T_3(n) = T_2(1) + T_2(2)+\dots + T_2(n)$
\begin{itemize}
    \item $T_3(5) = 1+3+6+10+15=35$
\end{itemize}
\addlink{Tetrahedral numbers}

\defn Simplex numbers: $T_{k+1}(n) = T_k(1) + \dots +T_k(n)$
\addlink{Simplex numbers}

\end{document} 

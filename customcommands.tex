% Some custom commands to make typing easier. Feel free to add to it


% quick homomorphism
\newcommand{\homo}[1]{\varphi (#1)}
% quick inverse
\newcommand{\inv}[1]{#1^{-1}}
% quick generating set
\newcommand{\genset}[1]{\langle #1 \rangle}
% split problem into multiple parts
\newcommand{\spl}{\rule{\textwidth}{0.4pt}}
% quick normal subgroup
\newcommand{\nsub}{\trianglelefteq}
% quick Z/nZ
\newcommand{\znz}[1]{\mathbb{Z\text{/#1}Z}}
% quick stirling numbers
% \newcommand{\squig}{\genfrac{\lbrace}{\rbrace}{0pt}{}}
\newcommand{\squig}[2]{\left\{{#1 \atop #2}\right\}}
% quick multichoose
\newcommand{\mchoose}[2]{\left({#1\choose #2}\right)}

% quick projective surface
\newcommand{\proj}[1]{\mathbb P^{#1}(\mathbb C)}
% quick elliptic curve
\newcommand{\ellip}{E(\mathbb{C})}
% quick ramification
\newcommand{\ram}[1]{e_{#1}}

% todo list
\usepackage[color=yellow]{todonotes}
\newcommand{\tbd}[1]{\todo[inline]{#1}
\addcontentsline{toc}{subsubsection}{TODO}}

% scaling tikz
\usepackage{adjustbox}

% format bullets
% \usepackage[shortlabels]{enumitem}

% Document inherent packages to be loaded
\usepackage{amsmath, amsfonts, amssymb, amsthm, graphicx, url, xcolor, enumerate, tcolorbox}

%geometry helps manage margins, among other things.
\usepackage[rmargin=2in,lmargin=1in,tmargin=1in,bmargin=1in]{geometry}
\newcommand{\sidenote}[1]{\marginpar{\footnotesize \begin{itemize}
    \item[$\leftarrow$]\raggedright #1
\end{itemize}}}

\setlength{\headheight}{16pt}
\setlength{\marginparwidth}{1.5in}
\makeatletter      % make title smaller   
\def\@maketitle{   % custom maketitle 
\begin{center}
    {\Large \@title} \\[0.1in] \@author \\ {\small \@date}
\end{center}  
}
\setcounter{secnumdepth}{0}

% new color for the hypers
\definecolor{hyperpurp}{RGB}{108,113,196}
\usepackage[colorlinks = true,
            linkcolor = hyperpurp,
            urlcolor  = hyperpurp,
            citecolor = hyperpurp,
            anchorcolor = hyperpurp]{hyperref}

\usepackage[all,2cell,ps]{xy}
\usepackage{tikz, tikz-cd}
\usepackage{faktor}
\allowdisplaybreaks
\graphicspath{{Images/}}

\theoremstyle{definition}  % to get rid of the italics
\newtheorem{theorem}{\bt{Theorem}}%[section]
\newtheorem{proposition}[theorem]{\bt{Proposition}}
\newtheorem{corollary}[theorem]{\gt{Corollary}}
\newtheorem{lemma}[theorem]{\bt{Lemma}}
\newtheorem{conjecture}[theorem]{Conjecture}

\newtheorem{defn}{\rt{Definition}}

\newtheorem{eg}{\textcolor{violet}{Example}}
\newtheorem{cautiouseg}[eg]{\textcolor{magenta}{Tricky example}}
\newtheorem{noneg}[eg]{\textcolor{violet}{Non-example}}

\newtheorem*{rmk}{Remark}


%Import the natbib package and sets a bibliography  and citation styles
\usepackage{natbib}

% Create headers
\usepackage{fancyhdr}
\usepackage{xcolor}
\renewcommand{\headrulewidth}{0pt}
\fancypagestyle{updated}{%
    \fancyhead{MATH103 Notes \hfill\hfill Xuehuai He}
    \fancyfoot{\hyperlink{toc}{Back to TOC}\hfill\thepage \hfill \textcolor{lightgray}{\today}}
}

% \newcommand{\Belyi}{Bely\u{\i}~}
% \newcommand{\thm}[3]{\vskip 0.2in \begin{center} \fbox{\begin{minipage}{0.8\linewidth} \begin{theorem}[#1] \label{#2} #3 \end{theorem} \end{minipage}} \end{center} \vskip 0.2in}
% \newcommand{\prop}[3]{\vskip 0.2in \begin{center} \fbox{\begin{minipage}{0.8\linewidth} \begin{proposition}[#1] \label{#2} #3 \end{proposition} \end{minipage}} \end{center} \vskip 0.2in}
% \newcommand{\cor}[3]{\vskip 0.2in \begin{center} \fbox{\begin{minipage}{0.8\linewidth} \begin{corollary}[#1] \label{#2} #3 \end{corollary} \end{minipage}} \end{center} \vskip 0.2in}
% \newcommand{\conj}[3]{\vskip 0.2in \begin{center} \fbox{\begin{minipage}{0.8\linewidth} \begin{conjecture}[#1] \label{#2} #3 \end{conjecture} \end{minipage}} \end{center} \vskip 0.2in}

\DeclareMathOperator{\Aut}{Aut}
\DeclareMathOperator{\Gal}{Gal}
\DeclareMathOperator{\Mon}{Mon}
\DeclareMathOperator{\lcm}{lcm}


\newcommand{\rt}[1]{\textcolor{magenta}{#1}}
\newcommand{\gt}[1]{\textcolor{teal}{#1}}
\newcommand{\bt}[1]{\textcolor{cyan}{#1}}
\newcommand{\pt}[1]{\textcolor{hyperpurp}{#1}}

\newcommand{\N}{\mathbb{N}}
\newcommand{\Z}{\mathbb{Z}}
\newcommand{\C}{\mathbb{C}}
\newcommand{\R}{\mathbb{R}}
\newcommand{\Q}{\mathbb{Q}}
\newcommand{\F}{\mathbb{F}}

\let\oldone\1
\renewcommand{\1}{\mathbb{1}}

\newcommand{\ifnif}{\textit{if and only if }}

\newcommand{\addlink}[1]{\addcontentsline{toc}{subsubsection}{#1}}

\newcommand{\circled}[1]{\begin{tikzpicture}[baseline=(word.base)]
\node[draw, rounded corners, text height=8pt, text depth=2pt, inner sep=2pt, outer sep=0pt, use as bounding box] (word) {#1};
\end{tikzpicture}
}

\newcommand{\ontangent}[1]{

    \spl

\textcolor{darkgray}{\textit{(Scratch work begins)}}
{#1}
\textcolor{darkgray}{\textit{(Scratch ends here)}}

\spl

}

\newcommand{\drawing}[2]{\begin{figure}[H]
    \centering
    \includegraphics[width=#1]{Images/#2}
\end{figure}}

\usepackage{ulem}
\usepackage{float}
\usepackage{libertinus}

% Make enumeration better
\usepackage{enumitem}
\setlist[itemize]{parsep=0pt,topsep=0pt}
\setlist[enumerate]{parsep=0pt,topsep=0pt}

% \definecolor{pomona}{HTML}{#0057b8}

\parskip = .2in
\parindent = 0in

% clever ref (load last)
\usepackage[capitalise]{cleveref}
\newcommand{\fref}{\cref}
\newcommand{\Fref}{\Cref}
\newcommand{\prettyref}{\cref}
\newcommand{\newrefformat}[2]{}

 %Cleveref definitions
\crefname{lemma}{Lemma}{Lemmas}
\crefname{thm}{Theorem}{Theorems}
\crefname{defn}{Definition}{Definitions}
\crefname{notn}{Notation}{Notations}
\crefname{const}{Construction}{Constructions}
\crefname{prop}{Proposition}{Propositions}
\crefname{rem}{Remark}{Remarks}
\crefname{cor}{Corollary}{Corollaries}
\crefname{equation}{Equation}{Equations}
\crefname{ex}{Example}{Examples}